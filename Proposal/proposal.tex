\documentclass[english]{article}
\usepackage[utf8]{inputenc}
\usepackage[T1]{fontenc}
\usepackage{babel}
\usepackage{amsmath}
\usepackage{amssymb}
\usepackage{graphicx}
\usepackage{fancyhdr}
\usepackage{enumerate}
\pagestyle{fancy}
\fancyhf{}
\renewcommand{\headrulewidth}{0pt}
\setlength{\headheight}{40pt} 

\begin{document}

\title{Project Proposal: Register Machine Interpreter}

\author{Wei-Lin Wu and Johnnie Chang}

\maketitle
\thispagestyle{fancy}


\section{Background}
In theoretical computer science, there have been many computational models proposed 
to study computability and complexity that are equivalent to one another in terms of computational power. 
Two best-known models are Turing machine and $\lambda$-calculus. 
\emph{Register machine}, a lesser known model is put forth to discuss the Halting Problem 
%A lesser-known third model called \emph{register machine} is used to discuss Halting Problem 
and other decidability problems in the book Mathematical Logic by H.-D.\ Ebbinghaus, J.\ Flum and W.\ Thomas \cite{ebbinghaus2013mathematical}.
\medskip\\
A register machine is associated with a fixed alphabet $\mathcal{A} = \{a_0, \ldots, a_r\}$ and has countably many registers, 
each of which can store a string over $\mathcal{A}$ of an arbitrary (finite) length.
\medskip\\
A program for a register machine with alphabet $\mathcal{A}$ is a finite nonempty sequence of instructions that each takes one of the forms below:
\begin{itemize}
\item $L$ LET $\mathrm{R}_i = \mathrm{R}_i + a_j$, where $L, i, j \in \mathbb{N}$ and $j \leq r$ (Add-instruction: Add the letter $a_j$ at the end of the string in register $mathrm{R}_i$);
\item $L$ LET $\mathrm{R}_i = \mathrm{R}_i - a_j$, where $L, i, j \in \mathbb{N}$ and $j \leq r$ (Subtract-instruction: If the string in register $\mathrm{R}_i$ ends with the letter $a_j$, delete this $a_j$; otherwise leave the word unchanged);
\item $L$ IF $\mathrm{R}_i = \epsilon$ THEN $L'$ ELSE $L_0$ OR \ldots OR $L_r$, for $L, i, L', L_0, \ldots , L_r \in \mathbb{N}$, where $\epsilon$ denotes the empty string (Jump-instruction: If register $\mathrm{R}_i$ contains the empty string go to instruction labelled $L'$; if the string in register $\mathrm{R}_i$ ends with $a_0$ (or $a_1, \ldots, a_r$, respectively) go to instruction labelled $L_0$ (or $L_1, \ldots, L_r$, respectively));
\item $L$ PRINT, for $L \in \mathbb{N}$ (Print-instruction: Print as output the string stored in register
$R_0$);
\item $L$ HALT, for $L \in \mathbb{N}$ (Halt-instruction: Halt).
\end{itemize}
In a program of length $n + 1$, the prefixing labels of instructions are $0, 1, 2, \ldots, n$, 
every jump instruction refers to labels no greater than $n$, and only the last instruction is a halt instruction.
%\medskip\\
\section{Goal and Motivation}
We have found this book an excellent reference for mathematical logic, 
and would like to write a register-machine interpreter in Haskell 
to run the programs given in this book for a deeper level of understanding 
and to experiment our ideas concerning the compuatioal model of register machine.
%
%
\section{Work Schedule and Risk Assessment}
\begin{enumerate}[Week 1.]
\setcounter{enumi}{3}
\item Background reading: \cite{ebbinghaus2013mathematical} Ch. 10
\item Background reading: \cite{ebbinghaus2013mathematical} Ch. 10
\item Implementaion of the interpreter
\item Implementaion of additional features and testing with examples in \cite{ebbinghaus2013mathematical}
\item Short paper; presentation preparation
\end{enumerate}

\bibliographystyle{IEEEtran}
%\bibliography{IEEEabrv,mybibfile}
\bibliography{proposal}
\end{document}