\documentclass[a4paper]{article}

%% Language and font encodings
\usepackage[english]{babel}
\usepackage[utf8x]{inputenc}
\usepackage[T1]{fontenc}

%% Sets page size and margins
\usepackage[a4paper,top=3cm,bottom=2cm,left=3cm,right=3cm,marginparwidth=1.75cm]{geometry}

%% Useful packages
\usepackage{amsmath}
\usepackage{graphicx}
\usepackage[colorinlistoftodos]{todonotes}
\usepackage[colorlinks=true, allcolors=blue]{hyperref}

%% User added packages
\usepackage{listings}
\usepackage{enumerate}

%% User defined
\newcommand{\sete}[1]{\{#1\}}
\newcommand{\seq}[1]{\langle#1\rangle}
\newcommand{\alphabet}{\mathcal{A}}
\newcommand{\reg}[1]{\mathrm{R}#1}
\newcommand{\lbl}[1]{\mathrm{L}#1}
\newcommand{\add}{\mathrel{+}=}
\newcommand{\sub}{\mathrel{-}=}

\title{Register Machine Interpreter Report}
\author{Wei-Lin Wu and Chung-Ning Chang}

\begin{document}
\maketitle

% \begin{abstract}
% Your abstract.
% \end{abstract}

\section{Introduction}

In theoretical computer science, there have been many computational models proposed 
to study computability and complexity that are equivalent to one another in terms of computational power. 
Two best-known models are \emph{Turing machine} and \emph{$\lambda$-calculus}. 
\emph{Register machine}, a lesser known model is put forth to discuss the Halting Problem and other decidability problems in the book Mathematical Logic by H.-D.\ Ebbinghaus, J.\ Flum and W.\ Thomas \cite{ebbinghaus2013mathematical}.
In this project we implement an interpreter of the Register Machine using Haskell, motivated by the simplicity and elegance of Register Machine and the intriguing process of implementing an imperative language in a functional language.
Our interpreter reads a plain text file that contains a Register Machine program, parses it into abstract syntax trees (ASTs), and interprets the program by evaluating the ASTs and display the output.
We also tested the interpreter with the program, \emph{reverse}, which accepts an input string from the user and displays it reversed. In addition, we created a special interpreter, OurBestWishes, for the program \emph{Happy}, which prints the message ``Happy Summer Break! $\string^$\_$\string^$y''.


\section{Definitions}

\subsection{Register Machines}

A \emph{register machine} is an abstract computational model consisting of the following:
\begin{itemize}
\item A nonempty alphabet $\alphabet = \sete{a_0, \ldots, a_r}$ of symbols (characters)
\item Countably many registers $\reg{0}, \reg{1}, \reg{2}, \ldots$,
each contains an arbitrarily long finite string of symbols and acts like a stack
\item A program
\item An infinitely long output tape
\end{itemize}

\subsection{Register Machine Programs and Instructions}

A \emph{program} is a finite sequence $\seq{I_0, I_1, \ldots, I_n}$ of instructions from
\medskip\\
\begin{tabular}{ll}
\multicolumn{2}{c}{\textbf{Instruction Set}} \cr
Operation & Format \cr\hline\hline
Add & LET $\reg{i} \add c$ \cr
Sub & LET $\reg{i} \sub c$ \cr
Jump & IFEMPTY $\reg{i}$ THEN $\lbl{\epsilon}$ ELSE $\lbl{0}$ OR $\lbl{1}$ \ldots OR $\lbl{r}$ \cr
Print & PRINT \cr
Halt & HALT \cr
\end{tabular}
\medskip\\
that satisfies the requirements:
\begin{itemize}
\item The last instruction, $I_n$, is the only Halt instruction in a program.
\item The instructions in a program are executed in the same order as they appear in the program, except when a Jump instruction designates the next instruction to execute.
\item Prior to execution of the program, every register contains the empty string, except possibly $\reg{0}$, which may contain the input string.\\
\end{itemize}

Below we provide a more detailed description of each instruction.
\begin{itemize}
\item[Add:] The Add instruction in a register machine is not an arithmetic add, but a push operation on a stack. When a symbol $c$ is added to a register $\reg{i}$, $c$ is unconditionally appended to the string contained in $\reg{i}$.
\item[Sub:] Analogously to Add, the Sub instruction is similar to a pop operation on a stack but with a key difference, that is, the pop only takes place if a condition is satisfied. When a symbol $c$ is subtracted from a register $\reg{i}$, if the last symbol of the current string in $\reg{i}$ is indeed $c$, it is popped (and discarded), otherwise nothing is done.
\item[Jump:] The Jump instruction is a branching instruction, with the condition that a specified register $\reg{i}$ is empty. The number of branches matches the size of the alphabet incremented by $1$. The first branch ($\lbl{\epsilon}$) is taken if the condition is met, i.e.\ if $\reg{i}$ is empty, and otherwise, depending on what the last symbol in $\reg{i}$ is, one of the remaining branches is taken (more precisely, the branch $\lbl{j}$ is taken if the last symbol of $\reg{i}$ is $a_j$).
\item[Print:] The Print instruction copies the current string in the register $\reg{0}$ and prints it to the output tape, after the current content of the output tape.
\item[Halt:] This instruction signals the end of the program. On the other hand, if the Halt instruction is never reached, the program does not terminate.
\end{itemize}

\subsection{Example}
Given an alphabet $\alphabet = \{\verb|`a'|, \verb|`b'|\}$, the input string \verb|``abbaa''| (i.e., the initial state of the register $\reg{0}$ is \verb|``abbaa''|), and the following program, reverse (in which each instruction is preceded by its label):
\medskip\\
\begin{tabular}{rl}
0 & IFEMPTY R0 THEN 7 ELSE 1 OR 4; \cr
1 & LET R0 $\sub$ \verb|`a'|; \cr
2 & LET R1 $\add$ \verb|`a'|; \cr
3 & IFEMPTY R0 THEN 7 ELSE 1 OR 4; \cr
4 & LET R0 $\sub$ \verb|`b'|; \cr
5 & LET R1 $\add$ \verb|`b'|; \cr
6 & IFEMPTY R0 THEN 7 ELSE 1 OR 4; \cr
7 & IFEMPTY R1 THEN 14 ELSE 8 OR 11; \cr
8 & LET R1 $\sub$ \verb|`a'|; \cr
9 & LET R2 $\add$ \verb|`a'|; \cr
10 & IFEMPTY R1 THEN 14 ELSE 8 OR 11; \cr
11 & LET R1 $\sub$ \verb|`b'|; \cr
12 & LET R2 $\add$ \verb|`b'|; \cr
13 & IFEMPTY R1 THEN 14 ELSE 8 OR 11; \cr
14 & IFEMPTY R2 THEN 21 ELSE 15 OR 18; \cr
15 & LET R2 $\sub$ \verb|`a'|; \cr
16 & LET R0 $\add$ \verb|`a'|; \cr
17 & IFEMPTY R2 THEN 21 ELSE 15 OR 18; \cr
18 & LET R2 $\sub$ \verb|`b'|; \cr
19 & LET R0 $\add$ \verb|`b'|; \cr
20 & IFEMPTY R2 THEN 21 ELSE 15 OR 18; \cr
21 & PRINT; \cr
22 & HALT \cr
\end{tabular}
\medskip\\
the output string would be \verb|``aabba''|, which is exactly the input string reversed. In fact, the algorithm for this program is
\begin{enumerate}[1.]
%
\item Move the string in $\reg{0}$ to $\reg{1}$ while reversing it (lines 0-6).
%
\item Move the string in $\reg{1}$ to $\reg{2}$ while reversing it (lines 7-13).
%
\item Move the string in $\reg{2}$ (back) to $\reg{0}$ while reversing it (lines 14-20).
%
\item Prints the string in $\reg{0}$ to the output tape (this print is not necessary) and halt (lines 21-22).
%
\end{enumerate}

% Comments can be added to your project by clicking on the comment icon in the toolbar above. % * <john.hammersley@gmail.com> 2016-07-03T09:54:16.211Z:
%
% Here's an example comment!
%
% To reply to a comment, simply click the reply button in the lower right corner of the comment, and you can close them when you're done.

% Comments can also be added to the margins of the compiled PDF using the todo command\todo{Here's a comment in the margin!}, as shown in the example on the right. You can also add inline comments:

% \todo[inline, color=green!40]{This is an inline comment.}

% \subsection{How to add Tables}

% Use the table and tabular commands for basic tables --- see Table~\ref{tab:widgets}, for example. 

% \begin{table}
% \centering
% \begin{tabular}{l|r}
% Item & Quantity \\\hline
% Widgets & 42 \\
% Gadgets & 13
% \end{tabular}
% \caption{\label{tab:widgets}An example table.}
% \end{table}

% \subsection{How to write Mathematics}

% \LaTeX{} is great at typesetting mathematics. Let $X_1, X_2, \ldots, X_n$ be a sequence of independent and identically distributed random variables with $\text{E}[X_i] = \mu$ and $\text{Var}[X_i] = \sigma^2 < \infty$, and let
% \[S_n = \frac{X_1 + X_2 + \cdots + X_n}{n}
%       = \frac{1}{n}\sum_{i}^{n} X_i\]
% denote their mean. Then as $n$ approaches infinity, the random variables $\sqrt{n}(S_n - \mu)$ converge in distribution to a normal $\mathcal{N}(0, \sigma^2)$.


% \subsection{How to create Sections and Subsections}

% Use section and subsections to organize your document. Simply use the section and subsection buttons in the toolbar to create them, and we'll handle all the formatting and numbering automatically.

% \subsection{How to add Lists}

% You can make lists with automatic numbering \dots

% \begin{enumerate}
% \item Like this,
% \item and like this.
% \end{enumerate}
% \dots or bullet points \dots
% \begin{itemize}
% \item Like this,
% \item and like this.
% \end{itemize}

% \subsection{How to add Citations and a References List}

% You can upload a \verb|.bib| file containing your BibTeX entries, created with JabRef; or import your \href{https://www.overleaf.com/blog/184}{Mendeley}, CiteULike or Zotero library as a \verb|.bib| file. You can then cite entries from it, like this: \cite{greenwade93}. Just remember to specify a bibliography style, as well as the filename of the \verb|.bib|.

% You can find a \href{https://www.overleaf.com/help/97-how-to-include-a-bibliography-using-bibtex}{video tutorial here} to learn more about BibTeX.

% We hope you find Overleaf useful, and please let us know if you have any feedback using the help menu above --- or use the contact form at \url{https://www.overleaf.com/contact}!

\section{Interpreter Implementation}

\subsection{Input Parsing}

\subsection{Abstract Syntax Trees}

\section{Conclusion}

\bibliographystyle{alpha}
\bibliography{proposal}

\end{document}